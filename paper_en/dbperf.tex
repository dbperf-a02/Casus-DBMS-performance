%==============================================================================
% Case research techniques: Database-performance
%==============================================================================
% Based on LaTeX-sjabloon ‘Stylish Article’ (cfr articletin.cls)
% Author: Jens Buysse, Bert Van Vreckem

% How to compile this document:
% 1) latexmk -pdf db-performance
% 2) biber db-performance
% 3) latexmk -pdf db-performance

\documentclass[fleqn,10pt]{articletin}

%------------------------------------------------------------------------------
% Metadata
%------------------------------------------------------------------------------

\JournalInfo{HoGent Business and Information Management} % Journal information
\Archive{Research 2016 - 2017} % Additional notes (e.g. copyright, DOI, review/research article)

%---------- Title, author- ----------------------------------------------------

\PaperTitle{Performance comparison between database systems}
\PaperType{Case research process} % Type document

\Authors{First name Name\textsuperscript{1}, First name Name\textsuperscript{2}, First name Name\textsuperscript{3}, First name Name\textsuperscript{4}} % Authors
\affiliation{\textbf{Contact:}
  \textsuperscript{1} \href{mailto:firstname.name@student.hogent.be}{firstname.name@student.hogent.be};
  \textsuperscript{2} \href{mailto:firstname.name@student.hogent.be}{firstname.name@student.hogent.be};
  \textsuperscript{3} \href{mailto:firstname.name@student.hogent.be}{firstname.name@student.hogent.be};
  \textsuperscript{4} \href{mailto:firstname.name@student.hogent.be}{firstname.name@student.hogent.be}}

%---------- Abstract ----------------------------------------------------------

  \Abstract{ This is where the summary of the paper goes, as a continuous text of one paragraph. This is actually the last thing to write, when the rest of the content is finished. The abstract should consist of the following elements: the \textbf{context} (why is this research important?), the \textbf{need} (why did it have to be researched?),the \textbf{task} (what exactly was done that is being reported on in this paper?), the \textbf{object} (what is written in this document?), the \textbf{result}, the most imporant \textbf{conclusion} and the \textbf{perspective} (what research could be conducted following this one?).  }


%---------- Research domain, keywords -----------------------------------------

\newcommand{\keywordname}{Keywords} % Defines the keywords heading name
\Keywords{Database-management --- Relational databases --- performance} % Keywords

%---------- Title, contents ---------------------------------------------------
\begin{document}

%\flushbottom % Makes all text pages the same height
\maketitle % Print the title and abstract box
\tableofcontents % Print the contents section
\thispagestyle{empty} % Removes page numbering from the first page

%------------------------------------------------------------------------------
% Body
%------------------------------------------------------------------------------

% The section division below is a suggestion that can be adjusted when
% necessary (e.g. split up experiments and analysis of results).

\section{Introduction} % The \section*{} command stops section numbering
\label{sec:introduction}

Describe the context of the conducted research, which also comprises the literature review, the need and task (see abstract). This section should have a lot of citations, at least to~\textcite{Bassil2012}.

A typical last sentence for an introduction is ``The rest of this article is structured as follows: Section~\ref{sec:methodology} describes the methodology that was followed, Section [\ldots]''

\section{Methodology}
\label{sec:methodology}

Describe in as much detail as possible how the experiment was designed. The reader should be able to reproduce your experiment independently using the description given.

\section{Experiments}
\label{sec:experiments}

Describe how the experiments went, and the most important results. Add graphs and tables that summarise/visualise the data.

Also describe the results of the hypothesis test: are the differences in performance significant?

\section{Conclusion}
\label{sec:conclusie}

Write down the conclusion(s) and any research questions that came up and that could be explored in future work.

%------------------------------------------------------------------------------
% Bibliography
%------------------------------------------------------------------------------

\phantomsection
\printbibliography[heading=bibintoc]

\end{document}
